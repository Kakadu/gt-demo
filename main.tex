%% For double-blind review submission, w/o CCS and ACM Reference (max submission space)
\documentclass[acmsmall,review,anonymous]{acmart}\settopmatter{printfolios=true,printccs=false,printacmref=false}
%% For double-blind review submission, w/ CCS and ACM Reference
%\documentclass[acmsmall,review,anonymous]{acmart}\settopmatter{printfolios=true}
%% For single-blind review submission, w/o CCS and ACM Reference (max submission space)
%\documentclass[acmsmall,review]{acmart}\settopmatter{printfolios=true,printccs=false,printacmref=false}
%% For single-blind review submission, w/ CCS and ACM Reference
%\documentclass[acmsmall,review]{acmart}\settopmatter{printfolios=true}
%% For final camera-ready submission, w/ required CCS and ACM Reference
%\documentclass[acmsmall]{acmart}\settopmatter{}


%% Journal information
%% Supplied to authors by publisher for camera-ready submission;
%% use defaults for review submission.
\acmJournal{PACMPL}
\acmVolume{1}
\acmNumber{CONF} % CONF = POPL or ICFP or OOPSLA
\acmArticle{1}
\acmYear{2017}
\acmMonth{1}
\acmDOI{} % \acmDOI{10.1145/nnnnnnn.nnnnnnn}
\startPage{1}

%% Copyright information
%% Supplied to authors (based on authors' rights management selection;
%% see authors.acm.org) by publisher for camera-ready submission;
%% use 'none' for review submission.
\setcopyright{none}
%\setcopyright{acmcopyright}
%\setcopyright{acmlicensed}
%\setcopyright{rightsretained}
%\copyrightyear{2017}           %% If different from \acmYear

%% Bibliography style
\bibliographystyle{ACM-Reference-Format}
%% Citation style
%% Note: author/year citations are required for papers published as an
%% issue of PACMPL.
\citestyle{acmauthoryear}   %% For author/year citations


%%%%%%%%%%%%%%%%%%%%%%%%%%%%%%%%%%%%%%%%%%%%%%%%%%%%%%%%%%%%%%%%%%%%%%
%% Note: Authors migrating a paper from PACMPL format to traditional
%% SIGPLAN proceedings format must update the '\documentclass' and
%% topmatter commands above; see 'acmart-sigplanproc-template.tex'.
%%%%%%%%%%%%%%%%%%%%%%%%%%%%%%%%%%%%%%%%%%%%%%%%%%%%%%%%%%%%%%%%%%%%%%


%% Some recommended packages.
\usepackage{booktabs}   %% For formal tables:
                        %% http://ctan.org/pkg/booktabs
\usepackage{subcaption} %% For complex figures with subfigures/subcaptions
                        %% http://ctan.org/pkg/subcaption

\usepackage{amsmath,amssymb}
\usepackage{listings}
\usepackage{amsmath,amssymb}

\def\backtick{\char18}

\lstdefinelanguage{ocaml}{
keywords={fresh, conde, let, begin, end, in, match, type, and, fun, function, try, with, when, class,
object, method, of, rec, repeat, until, while, not, do, done, as, val, inherit,
new, module, sig, deriving, datatype, struct, if, then, else, open, private, virtual, include, @type},
sensitive=true,
commentstyle=\small\itshape\ttfamily,
keywordstyle=\ttfamily\underbar,
identifierstyle=\ttfamily,
basewidth={0.5em,0.5em},
columns=fixed,
fontadjust=true,
literate={`}{$\backtick$}1 {->}{{$\to\;\;$}}3 {===}{{$\equiv$}}3 {=/=}{{$\not\equiv$}}3 {|>}{{$\triangleright$}}3,
morecomment=[s]{(*}{*)}
}

\lstset{
mathescape=true,
identifierstyle=\ttfamily,
keywordstyle=\bfseries,
commentstyle=\scriptsize\rmfamily,
basewidth={0.5em,0.5em},
fontadjust=true,
language=ocaml
}


\begin{document}

%% Title information
\title[Short Title]{Full Title}         %% [Short Title] is optional;
                                        %% when present, will be used in
                                        %% header instead of Full Title.
\titlenote{with title note}             %% \titlenote is optional;
                                        %% can be repeated if necessary;
                                        %% contents suppressed with 'anonymous'
\subtitle{Subtitle}                     %% \subtitle is optional
\subtitlenote{with subtitle note}       %% \subtitlenote is optional;
                                        %% can be repeated if necessary;
                                        %% contents suppressed with 'anonymous'


%% Author information
%% Contents and number of authors suppressed with 'anonymous'.
%% Each author should be introduced by \author, followed by
%% \authornote (optional), \orcid (optional), \affiliation, and
%% \email.
%% An author may have multiple affiliations and/or emails; repeat the
%% appropriate command.
%% Many elements are not rendered, but should be provided for metadata
%% extraction tools.

%% Author with single affiliation.
\author{First1 Last1}
\authornote{with author1 note}          %% \authornote is optional;
                                        %% can be repeated if necessary
\orcid{nnnn-nnnn-nnnn-nnnn}             %% \orcid is optional
\affiliation{
  \position{Position1}
  \department{Department1}              %% \department is recommended
  \institution{Institution1}            %% \institution is required
  \streetaddress{Street1 Address1}
  \city{City1}
  \state{State1}
  \postcode{Post-Code1}
  \country{Country1}                    %% \country is recommended
}
\email{first1.last1@inst1.edu}          %% \email is recommended

%% Author with two affiliations and emails.
\author{First2 Last2}
\authornote{with author2 note}          %% \authornote is optional;
                                        %% can be repeated if necessary
\orcid{nnnn-nnnn-nnnn-nnnn}             %% \orcid is optional
\affiliation{
  \position{Position2a}
  \department{Department2a}             %% \department is recommended
  \institution{Institution2a}           %% \institution is required
  \streetaddress{Street2a Address2a}
  \city{City2a}
  \state{State2a}
  \postcode{Post-Code2a}
  \country{Country2a}                   %% \country is recommended
}
\email{first2.last2@inst2a.com}         %% \email is recommended
\affiliation{
  \position{Position2b}
  \department{Department2b}             %% \department is recommended
  \institution{Institution2b}           %% \institution is required
  \streetaddress{Street3b Address2b}
  \city{City2b}
  \state{State2b}
  \postcode{Post-Code2b}
  \country{Country2b}                   %% \country is recommended
}
\email{first2.last2@inst2b.org}         %% \email is recommended


%% Abstract
%% Note: \begin{abstract}...\end{abstract} environment must come
%% before \maketitle command
\begin{abstract}
Text of abstract \ldots.
\end{abstract}


%% 2012 ACM Computing Classification System (CSS) concepts
%% Generate at 'http://dl.acm.org/ccs/ccs.cfm'.
\begin{CCSXML}
<ccs2012>
<concept>
<concept_id>10011007.10011006.10011008</concept_id>
<concept_desc>Software and its engineering~General programming languages</concept_desc>
<concept_significance>500</concept_significance>
</concept>
<concept>
<concept_id>10003456.10003457.10003521.10003525</concept_id>
<concept_desc>Social and professional topics~History of programming languages</concept_desc>
<concept_significance>300</concept_significance>
</concept>
</ccs2012>
\end{CCSXML}

\ccsdesc[500]{Software and its engineering~General programming languages}
\ccsdesc[300]{Social and professional topics~History of programming languages}
%% End of generated code


%% Keywords
%% comma separated list
\keywords{keyword1, keyword2, keyword3}  %% \keywords are mandatory in final camera-ready submission


%% \maketitle
%% Note: \maketitle command must come after title commands, author
%% commands, abstract environment, Computing Classification System
%% environment and commands, and keywords command.
\maketitle


%\section{Introduction}

%Text of paper \ldots


%% Acknowledgments
\begin{acks}                            %% acks environment is optional
                                        %% contents suppressed with 'anonymous'
  %% Commands \grantsponsor{<sponsorID>}{<name>}{<url>} and
  %% \grantnum[<url>]{<sponsorID>}{<number>} should be used to
  %% acknowledge financial support and will be used by metadata
  %% extraction tools.
  This material is based upon work supported by the
  \grantsponsor{GS100000001}{National Science
    Foundation}{http://dx.doi.org/10.13039/100000001} under Grant
  No.~\grantnum{GS100000001}{nnnnnnn} and Grant
  No.~\grantnum{GS100000001}{mmmmmmm}.  Any opinions, findings, and
  conclusions or recommendations expressed in this material are those
  of the author and do not necessarily reflect the views of the
  National Science Foundation.
\end{acks}

\section{Regularity and non-regularity}
In a regular datatype declaration, occurrences of the declared type on the right-hand
side  of  the  defining  equation  are  restricted  to  copies  of  the  left-hand  side,  so
the recursion  is "tail recursive"
\footnote{from http://www.cs.ox.ac.uk/richard.bird/online/BirdMeertens98Nested.pdf}.

For example the \lstinline{type ('a,'b) t = A | B of ('a, int) t} is non-regular and we need to use
\textit{polymorphic recursion} to iterate over it. Without explicit annotation the iteration
function will have the wrong type.
\begin{lstlisting}
# let rec iter: 'a 'b . ('a, 'b) t -> unit = function A -> () | B x -> iter x;;
val iter : ('a, 'b) t -> unit = <fun>
# let rec iter = function A -> () | B x -> iter x;;
val iter : ('a, int) t -> unit = <fun>
\end{lstlisting}
In that particular case unfolding can help to do what we want without explicit type annotations
\begin{lstlisting}
let rec iter = function A -> () | B x -> iter x;;
let iter = function A -> () | B x -> iter x;;
\end{lstlisting}

\section{Sorts of types}
$\overline{remark~how~to~write~upper~line}$

\[
\underbrace{u'-P(x)u^2-Q(x)u-R(x)}_{\text{=0, since~$u$ is a particular solution.}}
I_{2,(\underbrace{\scriptstyle3,3,3,3,3}_5)}
\]

\begin{lstlisting}
type ($a_1$, ..., $a_n$) typename =
  (* constructors of algebraic types *)
  | $C_1$ of $t_{11}$ * ... * $t_{1 n_1}$ | ... | $C_k$ of $t_{k1}$ * ... * $t_{k n_k}$
  (* tuples can be viewed as specific predefined constructors *)
  | ($t_1$, ..., $t_k$)
  (* polymorphic variants *)
  | [ `$C_1$ of $t_{11}$ * ... * $t_{1 n_1}$ | ... | `$C_k$ of $t_{k1}$ * ... * $t_{k n_k}$ ]
  (* type applications *)
  | ($t_1$, ..., $t_n$) othername

\end{lstlisting}

\section{What is being generated in general}

For every \lstinline{type ($a_1$, ..., $a_n$) t} we generate:
\begin{itemize}
  \item A class where methods correspond to constructor names. For normal and poly variants implementation is strightforward. For other sorts of types call will be populated by inheritance
  \begin{lstlisting}
  class virtual [ $class~arguments$ ] class_t = object
    (* either *)
      (* for every constructor (of normal or poly variant) called
         $C_i$ of $t_{i1}$ * ... * $t_{i n_1}$
      *)
      (* method virtual c_$C_i$: 'inh -> $t_{i1}$ -> ... ->  $t_{in}$ -> 'syn*)
      method virtual c_$C_i$: 'inh -> $curried~constructor~arguments$ -> 'syn
    (* or for type $\tau$ application to the args $t_1$ ... $t_n$ *)
      inherit [$with~application\!-\!specific~arguments$] class_$\tau$
  end
  \end{lstlisting}

  For every $n$-parametric type ($n>=0$) there are $3*n+2$ \textit{class arguments}. They are:
\begin{center}
    $\overline{a_i, sa_i, ia_i}$, $'inh$, $'syn$
\end{center}
  where \begin{itemize}
          \item $a_i$ is an $i$-th paramter of the type being processed
          \item $sa_i$ is a synthesized attribute for type parameter $a_i$
          \item $ia_i$ is an inherited attribute for type parameter $a_i$
          \item $'syn$ is a synthesized attribute for the whole type
  \end{itemize}
  \item A \textit{generic catamorphism} $gcata_t$ (for the type $t$ being processed) which will traverse the type and apply transformation. It is defined using pattern-matching for the variant types or as a composition with generic catamorphism $gcata_\tau$ if type $t$ is constructed as an application of type $\tau$.
  \item For every transformation method $tr$ a concrete class with transformation's implementation

  \begin{lstlisting}
  class [ $tr~class~arguments$ ] $tr$_t self $f\!a_1$ ... $f\!a_n$ = object
    inherit [ $inherited~class~arguments$ ] class_t
    (* implementation of virtual methods if any *)
  end
  \end{lstlisting}
  where \begin{itemize}
          \item transformation $self$ is a current transformation; class is defined in open recursion style and will receive it after tying the knot
          \item $f\!a_i$ of type \lstinline{$a_i$ -> $sa_i$} are transformation functions for type parameters
          \item $sa_i$ is a synthesized attribute for type parameter $a_i$
          \item $ia_i$ is an inherited attribute for type parameter $a_i$
          \item $'syn$ is a synthesized attribute for the whole type
  \end{itemize}
\bigskip
  TODO by Kakadu: I have strong feeling that $fa_i$ should have type \lstinline{$ia_i$ -> $a_i$ -> $sa_i$}. Motivating example will be a formatter plugin with \lstinline{'inh=== '$ia_i$ === Format.formatter} and \lstinline{'syn==='$sa_i$ === unit}.
  \item For every transformation method $tr$ a function which takes transformations $f\!a_i$ for every type parameter $a_i$, and a value of type $t$ that will be transformed
\end{itemize}

\section{xxx}

\section{xxx}
\begin{lstlisting}
   let rec append$^o$ x y xy =
     (x === nil ()) &&& (y === xy) |||
     (fresh (h t)
       (x === h % t) &&&
       (fresh (ty)
         (h % ty === xy) &&& (append$^o$ t y ty)
       )
     )

   let rec revers$^o$ a b =
     conde [
       (a === nil ()) &&& (b === nil ());
       (fresh (h t)
         (a === h % t) &&&
         (fresh (a')
           (append$^o$ a' !< h b) &&& (revers$^o$ t a')
         )
       )
     ]
\end{lstlisting}

%% Bibliography
%\bibliography{bibfile}

%% Appendix
%\appendix
%\section{Appendix}

Text of appendix \ldots


\end{document}
